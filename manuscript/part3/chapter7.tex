%!TEX root = ../thesis.tex
%*******************************************************************************
%********************************* Seventh Chapter *****************************
%*******************************************************************************
\cleardoublepage
\chapter{Application to wind turbine fatigue reliability and robustness}
\label{chpt:7}
%*******************************************************************************
\hfill
\localtableofcontents
\newpage




%============================================================%
%============================================================%
\section{Introduction}
%============================================================%
%============================================================%
One of the main goals of this work is to evaluate the reliability of an OWT's foundation with respect to fatigue solicitations. 
Let us recall the usual approach to assess fatigue damage over the structure's lifetime. 
First, the lifetime duration $T$ is discretized into $N_T \in \N$ 10-minutes intervals $\{t_i\}_{i=1}^{N_T}$. 
The 10-minutes duration results from the typical wind energy distribution (see \fig{fig:wind_psd}), presenting a ``short-term'' behavior (for turbulent wind with a return period below 10 minutes) and ``long-term'' behavior (otherwise).
Then the environmental conditions can be considered as a stochastic process $\{\bX(t^{(i)}, \omega^{(r)}), t^{(i)} \in T, \omega^{(r)} \in \Omega\}$ \elias{illustrated in \fig{fig:wind_long_short_term}}. 
For a realization $\bX(t^{(i)}, \omega^{(r)})$ and a given set of parameters $z$ related to the system (defined in Table~\ref{tab:sys_variables}), one can perform a 10-minutes DIEGO simulation and post-process it (see Section~\ref{sec:235}) to obtain a corresponding cumulative damage:
\begin{equation}
    d_{\mathrm{c}}^{10\mathrm{min.}}(\bX(t^{(i)}, \omega^{(r)})|Z=z). 
\end{equation}
To cumulate the damage over the lifetime, one can write the following sum of 10-minutes damages each averaged over $n_{\mathrm{rep}} \in \N$ pseudo-seeds repetitions: 
\begin{subequations}
    \begin{align}
        D(z) &= \frac{1}{n_{\mathrm{rep}}} \, \sum_{i=1}^{N_T} \sum_{r=1}^{n_{\mathrm{rep}}} d_{\mathrm{c}}^{10\mathrm{min.}}(\bX(t^{(i)}, \omega^{(r)})|Z=z)\\
             &= N_T \frac{1}{N_T \, n_{\mathrm{rep}}} \, \sum_{i=1}^{N_T} \sum_{r=1}^{n_{\mathrm{rep}}} d_{\mathrm{c}}^{10\mathrm{min.}}(\bX(t^{(i)}, \omega^{(r)})|Z=z)\\
             &\approx N_T \, \E_{\bX}\left[d_{\mathrm{c}}^{10\mathrm{min.}}(\bX|Z=z)\right],
             \label{eq:expected_damage}
    \end{align}
\end{subequations}
where $\bX \sim f_\bX$ is the random vector of the long-term environmental conditions, defined in Table~\ref{tab:envi_variables}. 

In Chapter~\ref{chpt:4}, kernel herding was proposed as a method for given-data sampling to propagate the uncertain environmental conditions on Teesside's model. 
This method showed equivalent performances as QMC for estimating the lifetime damage in \eq{eq:expected_damage}, while being more flexible. 
Considering the linear cumulative damage model (Miner's rule) used by the community, a damage value higher than one leads to fatigue failure by convention. 
The present chapter assesses the probability of such rare event, considering both the environmental uncertainties (aggregated according to \eq{eq:expected_damage}) and the uncertainties related to the system itself (described by the random variable $\bZ \in \iD_\bZ$ with PDF $f_\bZ$). 
Considering the $D_{\mathrm{cr}}$ the resistance variable centered around one, this failure probability is expressed as:
\begin{equation}
    \pf = \int_{\iD_\bZ} \1_{\{D(\bz) \geq D_{\mathrm{cr}}\}} \, f_\bZ(\bz) \, \dd z \,,
    \label{eq:owt_pf}
\end{equation}

Less information is available to define the probabilistic model of the system uncertainties than the environmental uncertainties. 
%For example, the soil geotechnical properties are by nature hard to measure without  
Therefore, the robustness of the failure probability to the probabilistic model $\bZ$ should be studied. 
To do so, a perturbation-based approach called the \textit{perturbed-law sensitivity indices} (PLI) is used in this chapter \citep{lemaitre_2015_PLI}.   

The present chapter is structured as follows: 
Section~\ref{sec:owt_surrogate} presents the construction of a surrogate model of $D(\cdot)$,
then Section~\ref{sec:owt_ra} analyses the reliability of the monopile foundation of Teesside's turbine for a nominal distribution of $\bZ$, 
and finally Section~\ref{sec:owt_robustness} proposes a robustness analysis of $\pf$ by perturbing the law of $\bZ$.

%============================================================%
%============================================================%
\section{Surrogate modeling for reliability analysis}\label{sec:owt_surrogate}
%============================================================%
%============================================================%
The prohibitive computational cost of the function $D(\cdot)$ requires fitting a surrogate model.  
This section presents the specific high-performance computer (HPC) wrapper developed for this application and its use to build a learning set for a Gaussian process regression. 

%============================================================%
\subsection{High-performance computer evaluation}
%============================================================%
The wrapper of the numerical chain including Turbsim and Diego (illustrated in xxx) for reliability analysis has a nested double loop structure.  
The outer loop pilots the realizations of $\bZ$ while the inner loop concerns the environmental conditions and their repetitions with $n_{\mathrm{rep}}$ different random seeds. 
At this stage, the goal is to approximate $\what{D}(k_{\mathrm{soil}}, \theta_{\mathrm{yaw}}):\R^2 \rightarrow \R$: 
\begin{equation}
    \what{D}(k_{\mathrm{soil}}, \theta_{\mathrm{yaw}}) = N_T \sum_{i=1}^{n_\bX} \sum_{r=1}^{n_{\mathrm{rep}}} d_{\mathrm{c}}^{10\mathrm{min.}}(\bx^{(i)}, \omega^{(r)} | K_{\mathrm{soil}}, \Theta_{\mathrm{yaw}}),
    \label{eq:hpc_cumulated_damage}
\end{equation}
where $\{\bx^{(i)}\}_{i=1}^{n_\bX}$ are defined by kernel herding, and $(K_{\mathrm{soil}}, \Theta_{\mathrm{yaw}})\TT$ are the system variables which are direct inputs of DIEGO (i.e., not part of the post-processing). 
According to the convergence results obtained in Chapter~\ref{chpt:4}, the kernel herding size is fixed at $n_\bX=200$ and the repetitions at $n_{\mathrm{rep}}=11$, which implies a total of $2200$ Turbsim-DIEGO simulations per evaluation of the function $\what{D}(\cdot)$. 
In this setup, the CRONOS HPC from EDF allows us to simultaneously perform those 2200 simulation in parallel. 
The random variable associated with the SN-curve uncertainty is introduced later as a product factor of \eq{eq:hpc_cumulated_damage}. 
\elias{Mention that this quantity is obtained for each azimuth angle of the muline node in the DIEGO model and the max is kept}

%============================================================%
\subsection{Design of experiments}
%============================================================%

To build a learning set, a space-filling design of experiments is built on the joint domain of $(K_{\mathrm{soil}}, \Theta_{\mathrm{yaw}})$. 
This design was first composed of 30 points, illustrated in \fig{fig:initial_doe}.a generated by a Halton sequence which was the most space-filling sequential method in two dimensions. 
Its evaluation and analysis showed that the highest damage values are the result of high yaw misalignment errors and low soil stiffness. 
The design was completed in a second phase by 20 points targeting these areas by applying a kernel herding in a subdomain defined a priori (see the candidate set represented by the gray points in \fig{fig:initial_doe}.b). 
Finally, the learning set is the union of the two complementary design which is also called ``composite design'', in reference to the heterogeneous composition of composite materials. 
This composite design, denoted by $\bZ_{n_\bZ}$, has a size of $n_\bZ=50$ which represents in total over $10^5$ Turbsim-DIEGO simulations (each requiring around 45 minutes of CPU time).
\fig{fig:evaluated_doe} shows the mean damage evaluated on the composite design corresponding to the normalized color scale. 


\elias{Comment on the fact that it's not symmetric}. 
\elias{the damage values are obtained for a nominal SN curve defined in xxxx for Teesside's model defined in Chapter 4}


\begin{figure}
    \centering
    \begin{subfigure}{0.48\linewidth}
        \vskip -30pt
        \includegraphics[width=\linewidth]{./part3/figures/OWT/initial_halton.png}
        \caption{Halton sequence.}
    \end{subfigure}
    \begin{subfigure}{0.48\linewidth}
        %\vskip 0pt
        \includegraphics[width=\linewidth]{./part3/figures/OWT/initial_composite.png}
        \caption{Composite design.}
    \end{subfigure}
    \caption{Learning set of the mean damage surrogate model. A Halton sequence is first built (in blue) and competed by kernel herding points (in orange) in a subdomain defined a priori (in gray).}
    \label{fig:initial_doe}
\end{figure}

\begin{figure}[h!]
    \centering
    \includegraphics[width=0.6\linewidth]{./part3/figures/OWT/normalized_results_mean.png}
    \caption{Normalized mean damage evaluated on the composite design illustrated in \fig{fig:initial_doe}.b.}
    \label{fig:evaluated_doe}
\end{figure}


%============================================================%
\subsection{Gaussian process regression}
%============================================================%

A Gaussian process regression with Matérn $5/2$ and constant trend is fitted on the composited design $\bZ_{n_\bZ}$ according to the kriging equation introduced in Section~\ref{sec:surrogate}. 
The resulting surrogate model, denoted by $\widetilde{D}:\R^2 \rightarrow \R$, is represented by the blue three-dimensional surface in \fig{fig:3d_owt_surrogate}, and its learning set  $\bZ_{n_\bZ}$ by the black crosses.  
A complementary visualization of this surrogate is proposed for fixed values: $k_{\mathrm{soil}}=1$ on \fig{fig:owt_surrogate}.b, and $\theta_{\mathrm{yaw}}=0$ on \fig{fig:owt_surrogate}.a. 
On these two figures, learning points are plotted in a gray scale and the surrogate model in a blue scale. 
The darkest the shade, the closest to the cross-section the points are. 

To validate this surrogate model, a leave-one-out (LOO) procedure is realized. 
\fig{fig:loo_validation}.b represents the LOO squared-residuals (corresponding to the color scale) at each points of the design. 
High residual values are mostly due to the strong nonlinearity of the code in some areas (as illustrated by the cross-section in \fig{fig:owt_surrogate}.b).
\fig{fig:loo_validation}.a shows the quantile-quantile plot comparing the LOO predictions with the mean damage evaluations on the learning set. 
A general coefficient of predictivity of $\what{Q}^2_{LOO}=0.72$ is considered acceptable in small data as the LOO procedure was shown to underestimate the performance metric in Chapter~\ref{chpt:5}. 
However, more points could be added to complete the learning set in regions with high yaw misalignment to enhance the surrogate presented.  

\begin{remark}
    ~
    \begin{itemize}
        \item Active learning methods for reliability (see Section~\ref{sec:active_surrogates}) could be a great option for such costly function. 
        However, the stochasticity of the function could disturb the learning criterions, and the enrichment area is well defined according the previous results. 
        \item Another approach would be to fit a stochastic surrogate \citep{binois_2019_replication,baker_2020_stochastic_surrogates_review,zhu_2023_thesis} on a learning set before averaging on the pseudo-random seed repetitions.  
    \end{itemize} 
\end{remark}

\begin{figure}[h!]
    \centering
    \includegraphics[width=0.52\linewidth]{./part3/figures/OWT/3D_surrogate.png}
    \caption{Three-dimensional plot of the surrogate model $\widetilde{D}$ (in blue) and learning set (in black).}
    \label{fig:3d_owt_surrogate}
\end{figure}

\begin{figure}[h!]
    \centering
    \begin{subfigure}[b]{0.38\linewidth}
        \includegraphics[width=\linewidth]{./part3/figures/OWT/dam_vs_soil_surrogate.png}
        \caption{$\theta_{\mathrm{yaw}}=0$.}
    \end{subfigure}
    \begin{subfigure}[b]{0.38\linewidth}
        \includegraphics[width=\linewidth]{./part3/figures/OWT/dam_vs_yaw_surrogate.png}
        \caption{$k_{\mathrm{soil}}=1$.}
    \end{subfigure}
    \caption{Cross-section of the surrogate model $\widetilde{D}$ (in shades of blue) for given values of $k_{\mathrm{soil}}$ and $\theta_{\mathrm{yaw}}$. The darkest the shade, the closest to the cross-section.}
    \label{fig:owt_surrogate}
\end{figure}

\begin{figure}[h!]
    \centering
    \begin{subfigure}[t]{0.36\linewidth}
        \includegraphics[width=\linewidth]{./part3/figures/OWT/loo_qqplot.png}
        \caption{Quantile-quantile plot.}
    \end{subfigure}
    \begin{subfigure}[t]{0.48\linewidth}
        \includegraphics[width=\linewidth]{./part3/figures/OWT/loo_squared_res.png}
        \caption{LOO Squared-residuals.}
    \end{subfigure}
    \caption{Leave-one-out validation results of the surrogate model $\widetilde{D}$.}
    \label{fig:loo_validation}
\end{figure}

%============================================================%
%============================================================%
\section{Reliability and robustness analysis}
%============================================================%
%============================================================%
In the wind energy industry, the acceptable risk levels for fatigue are defined by the standards. 
The target failure probability of the order of magnitude of $10^{-4}$ over the last year of exploitation is indicated by the \citet{iec_2019}. 
\citet{nielsen_2021_risk_levels} discusses the relevance of this risk level, defined from an economic point of view, and offers quantitative guidelines for lifetime extension. 
In this section, a probabilistic reliability analysis is conducted using the surrogate model of cumulative damage over the lifetime. 
Afterwards, the robustness of this reliability analysis is studied by applying perturbations to the input distributions and computing the perturbed law indices (PLI) initially proposed by \citet{lemaitre_2015_PLI}. 


%============================================================%
\subsection{Nominal reliability analysis}\label{sec:owt_ra}
%============================================================%

The surrogate model of the cumulative damage over lifetime $\widetilde{D}$ is first modified to include the SN curve uncertainty, defined in Section~\ref{sec:probabilistic_sn}:
\begin{equation}
    \widetilde{D}'(\bz) = \widetilde{D}'(k_{\mathrm{soil}}, \theta_{\mathrm{yaw}}, \varepsilon) = \frac1\varepsilon \, \widetilde{D}(k_{\mathrm{soil}}, \theta_{\mathrm{yaw}}).
\end{equation} 
The probability introduced in \eq{eq:owt_pf} then becomes: 
\begin{equation}
    \pf = \int_{\iD_\bZ} \1_{\{\widetilde{D}'(\bz) \geq D_{\mathrm{cr}}\}} \, f_\bZ(\bz) \, \dd z \,,
    \label{eq:owt_pf_surrogate}
\end{equation}
Table~\ref{tab:pf_result_table} presents the estimates of this quantity by different methods (FORM, FORM-importance sampling and subset simulation, see Section~\ref{sec:reliability}) and for two hypotheses regarding the distribution of the resistance variable. 
$D_{\mathrm{cr}}$ either follows a lognormal distribution (which has a short left tail), or a normal distribution (with a heavier left tail). 
All the methods deliver similar values of $\pf$ even if subset simulation requires more time. 
The adequation of FORM with the simulation methods reveals that the limit-state function in this case is almost linear. 
Such a conclusion might differ depending on the OWT model studied (e.g., a floating model might be different). 
The probabilities are, as expected, much lower under the hypothesis of lognormal distribution for $D_{\mathrm{cr}}$ than for a normal distribution. 
However, this significant difference questions the robustness of this result to the probabilistic model of $D_{\mathrm{cr}}$ and $\bZ$.  

\begin{table*}[h]
    \centering
    \caption{Nominal reliability analysis (IS and SS size $N=5 \times 10^4$, SS $p_0=0.1$).}
    \begin{tabular}{l||c|c|c|c}
              &  \multicolumn{2}{c|}{$D_{\mathrm{cr}} \sim $ \bf Lognormal} & \multicolumn{2}{c}{$D_{\mathrm{cr}} \sim $ \bf Normal}\\
    \hline
    \bf RA method & $\widehat{\pf}$       & $\widehat{\mathrm{cov}}$    & $\widehat{\pf}$       & $\widehat{\mathrm{cov}}$ \\
    \hline\hline
    FORM      & $9.87 \times 10^{-13}$ & --                    & $3.35 \times 10^{-6}$ & --\\
    \hline
    FORM-IS   & $9.84 \times 10^{-13}$ & $1 \%$                & $3.36 \times 10^{-6}$ & $1 \%$\\
    \hline
    SS        & $9.46 \times 10^{-13}$ & $7 \%$               & $3.50 \times 10^{-6}$ & $4 \%$\\ 
    \end{tabular}
    \label{tab:pf_result_table}
\end{table*}


%============================================================%
\subsection{Robustness analysis using the perturbed-law sensitivity indices}\label{sec:owt_robustness}
%============================================================%
The method of \citet{lemaitre_2015_PLI}, later called perturbed-law based indices (PLI) by \citet{sueur_2017_PLI} relies on perturbating densities. 
The goal is to assess the robustness of a quantity of interest (a failure probability in our case) to these perturbations. 
An application of the PLI on a thermal-hydraulic numerical code from the nuclear industry is proposed in \citet{iooss_2022_pli}. 


Assuming a random variable $Z_j \sim f_j \in \iD_{Z_j}$ with mean $\E[Z_j]=\mu$, variance $\var(Z_j) = \sigma^2$, the strategy is to find the ``closest'' distribution $f_{j \delta}$ under the constraint of moment perturbation.  
The notion of proximity between distributions is quantified by \citet{lemaitre_2015_PLI} terms of Kullback–Leibler divergence (KL). 
For example, a relative mean perturbation is defined as: 
\begin{eqnarray}
    f_{j \delta} = \argmin_{\substack{\pi\in \mathcal{P}, \\ \mathrm{s.t.}, \E_\pi[Z_j] = \E_{f_j}[Z_j](1 + \delta)}} \, \mathrm{KL}(\pi||f_j), \quad \delta \in \R. 
\end{eqnarray} 

Note that the perturbed distribution $f_{j \delta}$ might not belong to the parametric family of $f_j$. 
This i typically the case for bounded distributions, for example when perturbating the mean of a uniform distribution.  
To simplify the perturbation problem studied in \citet{lemaitre_2015_PLI, gauchy_2022_PLI}, the perturbations realized in the following will conserve the initial parametric family (which seems reasonable for distributions in the exponential family). 

The adapted expression of the PLI used hereafter \citep{gauchy_2022_PLI} reflects the relative impact of a pertubation on the quantity of interest: 
\begin{equation}
    \mathrm{PLI}(f_{j \delta}) = \frac{p_{\mathrm{f}, j \delta} - \pf }{\pf} \,,
\end{equation}
where $p_{\mathrm{f}, j \delta}$ is the probability obtained when injecting $f_{j \delta}$ in \eq{eq:owt_pf_surrogate}.

In our case, each variable in $\bZ$ is perturbed one by one in terms of relative standard deviation, such that $\sigma_{j \delta} = \sigma_j (1+\delta)$. 
See the illustration of such perturbations in \fig{fig:perturbations}, for distributions of $K_{\mathrm{soil}}$ on the left, and $\Theta_{\mathrm{yaw}}$ on the right.
This strategy assumes that the analyst has enough information to determine the mean of the variables $Z_j$. 
\fig{fig:pli_all} presents the PLI results obtained for relative perturbations of the standard deviations of $(K_{\mathrm{soil}}, \Theta_{\mathrm{yaw}}, \epsilon)$. 
The computation of the failure probabilities is done by independent FORM-IS estimations. 
In the hypothesis of a normal $D_{\mathrm{cr}}$, the most important variable is $\Theta_{\mathrm{yaw}}$, while the fluctuations are quite stable in the hypothesis of a lognormal $D_{\mathrm{cr}}$. 

When perturbating the standard deviation of the resistance variable $D_{\mathrm{cr}}$, the same phenomenon is witnessed in \fig{fig:pli_resistance}. 
The perturbations have nerly no consequences, assuming that $D_{\mathrm{cr}}\sim \mathrm{LogNormal}$, but a lot of influence when $D_{\mathrm{cr}}\sim \mathrm{Normal}$.  
As a perspective, this study could be completed by a joint perturbation of both standard deviation and mean of the resistance variable.


\begin{figure}
    \centering
        \includegraphics[width=0.43\linewidth]{./part3/figures/OWT/lognormal_pert.png}
        \includegraphics[width=0.44\linewidth]{./part3/figures/OWT/normal_pert.png}
    \caption{Perturbations in terms of standard deviation of a lognormal distribution (left) and a truncated normal distribution (right).}
    \label{fig:perturbations}
\end{figure}


\begin{figure}
    \centering
    \begin{subfigure}[t]{0.48\linewidth}
        \includegraphics[width=\linewidth]{./part3/figures/OWT/PLI_ALL_Hyp_LogNormal.png}
        \caption{$D_{\mathrm{cr}} \sim $ Lognormal.}
    \end{subfigure}
    \begin{subfigure}[t]{0.48\linewidth}
        \includegraphics[width=\linewidth]{./part3/figures/OWT/PLI_ALL_Hyp_Normal.png}
        \caption{$D_{\mathrm{cr}} \sim $ Normal.}
    \end{subfigure}
    \caption{Perturbed-law based indices for relative perturbation of the standard deviations of $(K_{\mathrm{soil}}, \Theta_{\mathrm{yaw}}, \epsilon)$. 
    The failure probabilities studied are each estimated by FORM-IS method with sample size $N=10^4$.}
    \label{fig:pli_all}
\end{figure}


\begin{figure}
    \centering
    \begin{subfigure}[t]{0.48\linewidth}
        \includegraphics[width=\linewidth]{./part3/figures/OWT/PLI_Dcr_Hyp_LogNormal.png}
        \caption{$D_{\mathrm{cr}} \sim $ Lognormal.}
    \end{subfigure}
    \begin{subfigure}[t]{0.45\linewidth}
        \includegraphics[width=\linewidth]{./part3/figures/OWT/PLI_Dcr_Hyp_Normal.png}
        \caption{$D_{\mathrm{cr}} \sim $ Normal.}
    \end{subfigure}
    \caption{Perturbed-law based indices for relative perturbation of the standard deviation of $D_{\mathrm{cr}}$. 
    The failure probabilities studied are each estimated by FORM-IS method with sample size $N=10^4$.}
    \label{fig:pli_resistance}
\end{figure}

\newpage
%============================================================%
%============================================================%
\section{Conclusion}
%============================================================%
%============================================================%


\begin{itemize}
    \item Turbines are stopped about 6pc of their lifetime, which reduces the aerodynamic amortissement and therefore increases fatigue
    \item Starts and stops increase the damage
    \item Installation of the foundation can create early damage
    \item Stress concentration factors were fixed to 1
\end{itemize}


