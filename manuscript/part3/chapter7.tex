%!TEX root = ../thesis.tex
%*******************************************************************************
%********************************* Seventh Chapter *****************************
%*******************************************************************************
\cleardoublepage
\chapter{Application to wind turbine fatigue reliability and robustness}
\label{chpt:7}
%*******************************************************************************
\hfill
\localtableofcontents
\newpage

On risk levels for OWT: https://onlinelibrary.wiley.com/doi/epdf/10.1002/we.2610


%============================================================%
%============================================================%
\section{Introduction}
%============================================================%
%============================================================%
\begin{itemize}
    \item Recall the system random variables
    \item Define the specific QoI, maybe with a diagram
\end{itemize}




%============================================================%
%============================================================%
\section{Surrogate modeling for reliability analysis}
%============================================================%
%============================================================%

\begin{itemize}
    \item Present the HPC implementation required to realize such computations
    \item Hybrid DoE construction (benefitting from the flexibility of the KH)
    \item Gaussian model validation made difficult by the strong nonlinearities
    \item Possible lack of trust in the mean estimation
\end{itemize}

\begin{itemize}
    \item Turbines are stopped about 6pc of their lifetime, which reduces the aerodynamic amortissement and therefore increases fatigue
    \item Starts and stops increase the damage
    \item Installation of the foundation can create early damage
\end{itemize}

%============================================================%
%============================================================%
\section{Reliability and robustness analysis}
%============================================================%
%============================================================%


%============================================================%
\subsection{Reliability analysis}
%============================================================%
\begin{itemize}
    \item Nominal reliability analysis for 3 possible models of $D_cr$ and using multiple RA methods including BANCS
\end{itemize}

%============================================================%
\subsection{Robustness analysis}
%============================================================%
\begin{itemize}
    \item Definition of the PLI
    \item Perturbation protocol (only on the variance)
    \item Numerical results and discussion on the relevance of the study  
\end{itemize}


%============================================================%
%============================================================%
\section{Conclusion}
%============================================================%
%============================================================%




