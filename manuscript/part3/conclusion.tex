%!TEX root = ../thesis.tex
%*******************************************************************************
%*********************************** Conclusion Chapter *****************************
%*******************************************************************************
\chapter*{Conclusion and perspectives}
\addcontentsline{toc}{chapter}{Conclusion and perspectives} 
\chaptermark{Conclusion and perspectives}
%*******************************************************************************


%============================================================%
%============================================================%
\section*{Summary of the main contributions and perspectives}
%============================================================%
%============================================================%
Uncertainty quantification on the industrial application of offshore wind turbine (OWT) raises numerous methodological questions. 
In the present thesis, two main tools were exploited in very different contexts: the maximum mean discrepancy (MMD) as a kernel-based dissimilarity measure, and the empirical Bernstein copula (EBC) for multivariate nonparametric inference. 
This work first tackled problems related to uncertainty quantification of offshore environmental conditions (i.e., metocean), including their perturbation induced by the turbine's wake. 
In a second part, it dealt with uncertainty propagation for different goals (i.e., quantities of interest) such as mean estimation, mean assessment of a surrogate's predictivity, or rare event probability estimation.
For reproducibility purposes, most numerical developments related to this work are documented and openly accessible. 


%============================================================%
\subsection*{Uncertainty quantification of environmental conditions}
%============================================================%
%A SIMULATION STUDY ON THE USEFULNESS OF THE BERNSTEIN COPULA FOR STATISTICAL MODELING OF METOCEAN VARIABLES
\elias{add perspective refs}
The first contribution to this topic is a study on the efficiency of a semiparametric inference of the metocean data. 
In the proposed strategy, the multivariate copula is approximated by EBC, while the marginals can either be fitted by parametric or nonparametric method.   
The simplicity and high flexibility of the EBC suits the complex dependence structures found in metocean datasets. 
As perspectives, a more extensive numerical comparison (e.g., with Vine copula) could be performed, for example using the MMD as a statistic for multivariate testing. 
Note that the MMD could also serve as a metric to optimize the Bernstein polynomial order. 
From a wider perspective, goodness-of-fit for nonparametric methods might be studied in the same fashion as the validation of machine learning models (by wisely choosing a set of points to test the model).
Finally, this method could also be studied in the context of extreme conditions assessment, which is usually in favor of parametric approaches. 

After defining a probabilistic model of the ambient metocean conditions, the influence of the turbines' wake was studied. 
An uncertainty propagation of the ambient conditions through a simplified wake model of a wind farm provided the wake-perturbed metocean distribution at each turbine.  
Then our main contribution was to propose the MMD as a dissimilarity measure between these distributions. 
A matrix of dissimilarities was obtained by computing the MMD between every turbine, on which a simple clustering method allowed to gather turbines with similar conditions.  
By creating clusters of turbines solicited with similar environmental conditions, this approach aims at reducing the number loading studies at the farm scale (e.g., fatigue assessment).
These contributions are related to the following publications:
\begin{itemize}
    \footnotesize
    \item[\ding{125}] A. Lovera, \underline{E. Fekhari}, B. Jézéquel, M. Dupoiron, M. Guiton and E. Ardillon (2023). ``Quantifying and clustering the wake-induced perturbations within a wind farm for load analysis". In: \textit{Journal of Physics: Conference Series (WAKE 2023)}.
    \item[\ding{125}] E. Vanem, \underline{E. Fekhari}, N. Dimitrov, M. Kelly, A. Cousin and M. Guiton (2023). ``A joint probability distribution model for multivariate wind and wave conditions''. In: \textit{Proceedings of the ASME 2023 42th International Conference on Ocean, Offshore and Arctic Engineering (OMAE 2023)}. 
    \item[\ding{125}] E. Vanem, \O{}. Lande and \underline{E. Fekhari}, (2024). ``A simulation study on the usefulness of the Bernstein copula for statistical modeling of metocean variables''. . To appear in: \textit{Proceedings of the ASME 2024 43th International Conference on Ocean, Offshore and Arctic Engineering (OMAE 2024)}.
\end{itemize}

%============================================================%
\subsection*{Reliability and robustness analysis of offshore wind turbines subject to fatigue damage}
%============================================================%

These contributions are linked with uncertainty propagation on the multi-physics numerical model of an offshore wind turbine. 
A first contribution concerns the estimation of lifetime cumulative fatigue damage as a mean against the environmental conditions. 
The use of Bayesian quadrature methods such as the kernel herding is an efficient and flexible solution for given-data uncertainty propagation (i.e., directly subsampling from a large dataset without inference). 
Our numerical experiments showed conclusive results in comparison to Monte Carlo and quasi-Monte Carlo sampling. 



\elias{Reliability analysis over the system variables}
\elias{Reliability analysis, using the PLI method, due to the lack of information on these uncertainties}



\begin{itemize}
    \item Kernel herding is an efficient and flexible solution for given-data uncertainty propagation. 
    \item Application to the mean lifetime fatigue damage estimation and conclusive comparison with MC and QMC
    \item Reliability analysis considering random system variables (probabilistic SN curve, yaw misalignment, soil stiffness)
    \item Robustness analysis wrt the system variables and the random critical damage (add main conclusions)
\end{itemize}

\begin{itemize}
    \item Explore and adapt kernel herding for other quantities (e.g., quantiles) using conditional kernel mean embedding and potentially randomization. 
    \item Extend the reliability analysis on a floating OWT case, with a more challenging wave condition impact
    \item Assess the reliability of the Teesside wind farm by exploiting the representative wind turbines in terms of wake
    \item Explore nonlinear cumulative models 
    \item Connect this work with ULS conclusions: how to determine what is the most influential
\end{itemize}

These contributions are related to the following publications:
\begin{itemize}
    \footnotesize
    \item[\ding{125}] E. Fekhari, V. Chabridon, J. Muré and B. Iooss (2023). ``Given-data probabilistic fatigue assessment for offshore wind turbines using Bayesian quadrature''. In: \textit{Data-Centric Engineering}, In press.
    \item[\ding{125}] E. Fekhari, B. Iooss, V. Chabridon, J. Muré (2022). ``Efficient techniques for fast uncertainty propagation in an offshore wind turbine multi-physics simulation tool''. In: \textit{Proceedings of the 5th International Conference on Renewable Energies Offshore (RENEW 2022)}.
\end{itemize}

%============================================================%
\subsection*{Surrogate model predictivity assessment}
%============================================================%

\begin{itemize}
    \item Sequential test set construction and optimal weighting method for coefficients of predictivity
    \item Two contexts: ML (give-data) or computer experiments
\end{itemize}

\begin{itemize}
    \item stopping rule based on MMD
    \item Connect the validation with results from SA to give more importance to certain variables
\end{itemize}

This contribution is related to the following publication:
\begin{itemize}
    \footnotesize
    \item[\ding{125}] E. Fekhari, B. Iooss, J. Muré, L. Pronzato and M.J. Rendas (2023). ``Model predictivity assessment: incremental test-set selection and accuracy evaluation''. In: \textit{Studies in Theoretical and Applied Statistics}, pages 315--347. Springer.
\end{itemize}

%============================================================%
\subsection*{Adaptive rare event estimation using Bernstein copula}
%============================================================%

\begin{itemize}
    \item New algorithm for rare event estimation: BANCS
    \item Free ROSA as a post-processing of the iid samples from the RA. 
    The same way FORM delivers local information
\end{itemize}

\begin{itemize}
    \item Other sampling methods to improve the estimation
    \item Optimal Berstein copula estimation?
    \item Semi-parametric modeling?
    \item On high dimensional problems: exploit ROSA to simplify certain problems
    \item On high dimensional problems: EBC by block 
\end{itemize}

This contribution is related to the following publication:
\begin{itemize}
    \footnotesize
    \item[\ding{125}] E. Fekhari, V. Chabridon, J. Muré and B. Iooss (2023). ``Bernstein adaptive nonparametric conditional sampling: a new method for rare event probability estimation''. In: \textit{Proceedings of the 13th International Conference on Applications of Statistics and Probability in Civil Engineering (ICASP 14)}.
\end{itemize}




%============================================================%
%============================================================%
\section*{Summary of the numerical developments}
%============================================================%
%============================================================%

%============================================================%
\subsection*{Python package \texttt{otkerneldesign}}
%============================================================%
\begin{itemize}
    \item CUDA 
    \item Exploit the work on estimators realized in \ot
\end{itemize}


%============================================================%
\subsection*{Python package \texttt{bancs}}
%============================================================%

\begin{itemize}
    \item Documentation and packaging
    \item Perform an exhaustive benchmark
\end{itemize}

%============================================================%
\subsection*{Python package \texttt{copulogram}}
%============================================================%

\begin{itemize}
    \item Documentation
\end{itemize}

%============================================================%
\subsection*{Offshore wind turbine reliability wrapper}
%============================================================%

\begin{itemize}
    \item Adapt the current reliability wrapper to a floating case
\end{itemize}