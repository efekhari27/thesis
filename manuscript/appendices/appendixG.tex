%!TEX root = ../thesis.tex
% ******************************* Thesis Appendix G ****************************
\chapter{Uncertainty quantification implemented with \ot}
\label{apx:G}
%*******************************************************************************


\elias{Add short introduction to the motivation}

%%%%%%%%%%%%%%%%%%%%%%%%%%%%%%%%%%%%%%%%%%%%%%%%%%%%%%%%%%%%%%
\begin{otexample}[Bivariate distribution]
    The following Python code proposes a minimalistic \ot implementation of a probabilistic uncertainty modeling. 
    \lstset{style=mystyle, language=python}
\begin{lstlisting}
import openturns as ot
# Build multivariate distribution from marginals and copula
copula=ot.GumbelCopula(2.0)
marginals=[ot.Uniform(1.0, 2.0), ot.Normal(2.0, 3.0)]
distribution=ot.ComposedDistribution(marginals, copula)
# Compute first moments
mean_vector=distribution.getMean()
covariance_matrix=distribution.getCovariance()
# Compute CDF (respectively PDF)
x_cdf=distribution.computeCDF([1.5, 2.5])# x=[1.5, 2.5]
a_quantile=distribution.computeQuantile([0.9])# alpha=0.9
\end{lstlisting}
\end{otexample}
%%%%%%%%%%%%%%%%%%%%%%%%%%%%%%%%%%%%%%%%%%%%%%%%%%%%%%%%%%%%%%

%%%%%%%%%%%%%%%%%%%%%%%%%%%%%%%%%%%%%%%%%%%%%%%%%%%%%%%%%%%%%%
\begin{otexample}[Numerical integration]
    The following Python code proposes a minimalistic \ot implementation to build a quadrature rule. 
\lstset{style=mystyle, language=python}
\begin{lstlisting}
import openturns as ot
marginals=[ot.Exponential(1.0), ot.Uniform(-1.0, 1.0)]
distribution=ot.ComposedDistribution(marginals)
# Build a 2D Gaussian quadrature
n_marginal=[4, 4] # Number of nodes per marginal
g_quad=ot.GaussProductExperiment(distribution, n_marginal)
g_nodes, weights=g_quad.generateWithWeights()
# Build a Monte Carlo design
n=16 
mc_nodes=distribution.getSample(n)
# Build a quasi-Monte Carlo design
sequence=ot.HaltonSequence(2) # d=2
qmc_experiment=ot.LowDiscrepancyExperiment(sequence, distribution, n)
qmc_nodes=qmc_experiment.generate()
\end{lstlisting}
\end{otexample}
%%%%%%%%%%%%%%%%%%%%%%%%%%%%%%%%%%%%%%%%%%%%%%%%%%%%%%%%%%%%%%


%%%%%%%%%%%%%%%%%%%%%%%%%%%%%%%%%%%%%%%%%%%%%%%%%%%%%%%%%%%%%%
\begin{otexample}[Design of experiments]
    The following Python code proposes a minimalistic \ot implementation to build a LHS and 
    a LHS optimized w.r.t. to a space-filling metric (here the L2-centered discrepancy) using the simulated annealing algorithm. 
    \lstset{style=mystyle, language=python}
\begin{lstlisting}
import openturns as ot
marginals=[ot.Uniform(0.0, 1.0), ot.Uniform(0.0, 1.0)]
distribution=ot.ComposedDistribution(marginals)
# Build a LHS
n=10
LHS_exp=ot.LHSExperiment(distribution, n)
LHS_design=LHS_exp.generate()
# Build an optimized LHS using L2-centered discrepancy
LHS_exp=ot.LHSExperiment(distribution, n)
SF_metric=ot.SpaceFillingC2()
SA_profile=ot.GeometricProfile(10., 0.95, 20000)
LHS_opt=ot.SimulatedAnnealingLHS(LHS_exp, SF_metric, SA_profile)
LHS_opt.generate()
LHS_design=LHS_opt.getResult().getOptimalDesign()
\end{lstlisting}
\end{otexample}
%%%%%%%%%%%%%%%%%%%%%%%%%%%%%%%%%%%%%%%%%%%%%%%%%%%%%%%%%%%%%%


%%%%%%%%%%%%%%%%%%%%%%%%%%%%%%%%%%%%%%%%%%%%%%%%%%%%%%%%%%%%%%
\begin{otexample}[Rare event estimation]
    The following Python code proposes a minimalistic \ot implementation of rare event estimation algorithms. 
    \lstset{style=mystyle, language=python}
\begin{lstlisting}
import openturns as ot
marginals=[ot.Normal(0.0, 1.0), ot.Exponential(1.0)]
distribution=ot.ComposedDistribution(marginals)
# Build a limit-state function and failure event
g=ot.SymbolicFunction(["x1", "x2"], ["(x1 - x2) ^ 2"])
X=ot.RandomVector(distribution)
Y=ot.CompositeRandomVector(g, X)
th=0.0
failure_event=ot.ThresholdEvent(Y, ot.LessOrEqual(), th)
# Estimate pf using FORM
starting_p=distribution.getMean()
FORM_algo=ot.FORM(ot.Cobyla(), failure_event, starting_p)
FORM_algo.run()
FORM_results=FORM_algo.getResult()
design_point=FORM_results.getStandardSpaceDesignPoint()
FORM_pf=FORM_results.getEventProbability()
# Estimate pf using Monte Carlo 
MC_exp=ot.MonteCarloExperiment()
MC_algo=ot.ProbabilitySimulationAlgorithm(failure_event, MC_exp)
MC_algo.run()
MC_results=MC_algo.getResult()
MC_pf=MC_results.getProbabilityEstimate()
MC_pf_confidence=MC_results.getConfidenceLength(0.95)
# Estimate pf using importance sampling
aux_distribution=ot.Normal(design_point, [1.0, 1.0])
standard_event=ot.StandardEvent(failure_event)
IS_exp=ot.ImportanceSamplingExperiment(aux_distribution)
IS_algo=ot.ProbabilitySimulationAlgorithm(standard_event, IS_exp)
IS_algo.run()
IS_results=IS_algo.getResult()
IS_pf=IS_results.getProbabilityEstimate()
IS_pf_confidence=IS_results.getConfidenceLength(0.95)
# Estimate pf using subset sampling
SS_algo=ot.SubsetSampling(failure_event)
SS_algo.run()
SS_results=SS_algo.getResult()
SS_pf=SS_results.getProbabilityEstimate()
SS_pf_confidence=SS_results.getConfidenceLength(0.95)
\end{lstlisting}
\end{otexample}
%%%%%%%%%%%%%%%%%%%%%%%%%%%%%%%%%%%%%%%%%%%%%%%%%%%%%%%%%%%%%%


%%%%%%%%%%%%%%%%%%%%%%%%%%%%%%%%%%%%%%%%%%%%%%%%%%%%%%%%%%%%%%
\begin{otexample}[Ordinary kriging]
    The following Python code proposes a minimalistic \ot implementation of an ordinary kriging model fitting. 
    \lstset{style=mystyle, language=python}
\begin{lstlisting}
g=ot.SymbolicFunction(['x'], ['x * sin(x) + sin(6 * x)'])
x_train=ot.Uniform(0., 12.).getSample(7) # n=7
y_train=g(x_train)
basis=ot.ConstantBasisFactory(1).build() # d=1
cov_model=ot.MaternModel([1.], 1.5)
algo=ot.KrigingAlgorithm(x_train,y_train,cov_model,basis)
algo.run()
kriging_results=algo.getResult()
kriging_predictor=kriging_results.getMetaModel()
\end{lstlisting}
\end{otexample}
%%%%%%%%%%%%%%%%%%%%%%%%%%%%%%%%%%%%%%%%%%%%%%%%%%%%%%%%%%%%%%