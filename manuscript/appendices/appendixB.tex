%!TEX root = ../thesis.tex
% ******************************* Thesis Appendix B ****************************
\cleardoublepage
\chapter{Dissimilarity measures between probability distributions}
\label{apx:B}
%*******************************************************************************

Beyond the discrepancy measure to the uniform distribution, this section introduces different dissimilarity measures between probability distributions. 

%============================================================%
%============================================================%
\section{Csiz\'{a}r $f$-divergences}
%============================================================%
%============================================================%

\elias{General definition}

\elias{Numerous examples depending on the function choosen: see the book culte}

\elias{Link between KL and mutual information}
\elias{Further inputs in the review from Rahman, maybe some in the PhD subject from A.Dutfoy.}
\elias{Problems generated in the estimation}

%============================================================%
%============================================================%
\section{Integral probability metrics}
%============================================================%
%============================================================%

\elias{general definition}

\elias{Numerous examples see the book culte}

\elias{No closed form expression unlike the $f$-divergence but the use of RKHS goes around this issue.}


%============================================================%
\subsection*{Maximum discrepancy measure}
%============================================================%

A metric of discrepancy between distributions is introduced as the \emph{maximum mean discrepancy} (MMD). 
This distance between two probability distributions $\mu$ and $\zeta$ is defined as the worst-case error for any function within a unit ball of a function space $\iH$:
\begin{equation}
    \MMD(\mu, \zeta) := %\\ 
    \sup_{\lVert g \lVert_{\iH} \leq 1}
            \left | \int_{\iD_{\bX}} g(\bx) \dd \mu(\bx) - \int_{\iD_{\bX}} g(\bx) \dd \zeta(\bx) \right| = \left\lVert P_{\mu}(\bx) - P_{\zeta}(\bx) \right\lVert_{\iH}.
    \label{eq:mmd}  
\end{equation}

To ease the calculation of the quantity, this metric was studied for a particular function space, offering specific properties.
A \emph{reproducing kernel Hilbert space} (RKHS), denoted $\iH(k)$, is an inner product space $\iH(k)$ of functions $g:\iD_{\bX} \rightarrow \R$.
It verifies the following axioms, considering a symmetric and positive definite function $k: \iD_{\bX} \times \iD_{\bX} \rightarrow \R$, later called a ``reproducing kernel'' or simply a ``kernel'': 
\begin{itemize}
    \item The ``feature map'' $\phi : \iD_{\bX} \to \iH(k); \phi(\bx) = k(\cdot, \bx) \in \iH(k), \forall \bx \in \iD_{\bX}$.
    \item The ``reproducing property'': $\langle g, k(\cdot, \bx) \rangle_{\iH(k)} = g(\bx), \quad \forall \bx \in \iD_{\bX}, \forall g \in \iH(k)$.
\end{itemize}
Every positive semi-definite kernel defines a unique RKHS (and vice versa) with a feature map $\phi$, such that $k(\bx, \bx') = \langle \phi(\bx), \phi(\bx') \rangle_{\iH(k)}$.
Moreover, \cite{sriperumbudur_2010} defines a kernel as ``characteristic kernel'' when the following equivalence is true: $\MMD_k(\mu, \zeta) = 0 \Leftrightarrow \mu = \zeta$. 
This property makes the MMD a metric on $\iD_{\bX}$.

Then, a probability measure has a representation in the RKHS through its \emph{kernel mean embedding} \citep{sejdinovic_2013}, also called ``potential'' $P_{\mu}(\bx)$ in \cite{pronzato_zhigljavsky_2020}, defined as:
\begin{equation}
   P_{\mu}(\bx) := \int_{\iD_{\bX}} k(\bx, \bx') \dd \mu(\bx').
\end{equation}
The reproducing property from the RKHS allows to express the squared MMD as expectations of kernels:
\begin{equation}\label{eq:mmd2}
    \MMD_k(\mu, \zeta)^2 = \int_{\iD_{\bX}} P_{\mu}(\bx)\, \dd\mu(\bx) - 2 \int_{\iD_{\bX}} P_{\mu}(\bx) \, \dd\zeta(\bx) + \int_{\iD_{\bX}} P_{\zeta}(\bx)\, \dd\zeta(\bx).
\end{equation}

\elias{Add a sentence on estimation}
