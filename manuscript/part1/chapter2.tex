%!TEX root = ../thesis.tex
%*******************************************************************************
%*********************************** Second Chapter *****************************
%*******************************************************************************
\chapter{Introduction to wind turbine modeling and design}
%*******************************************************************************
\hfill
\localtableofcontents
\newpage



%https://eolienne.f4jr.org/eolienne_etude_theorique

%https://www.geotechnique-journal.org/articles/geotech/pdf/2018/04/geotech190004s.pdf

%https://en.wikipedia.org/wiki/Wind-turbine_aerodynamics

%@book{hansen2015aerodynamics,
%  title={Aerodynamics of wind turbines},
%  author={Hansen, Martin OL},
%  year={2015},
%  publisher={Routledge}
%}

%============================================================%
%============================================================%
\section{Introduction}
%============================================================%
%============================================================%

Wind energy is a highly competitive industry with increasing regulations regarding its impact on ecosystems, land-use conflicts, landscapes, or air traffic management \citep{eolien_en_mer_2022}. 
During the long process to win calls for tenders and obtain construction permits or throughout the wind farm exploitation, an advanced technical understanding of such systems might offer a competitive advantage. 

The operation of offshore wind turbines are driven by multiple physics coupled. 
This behavior results from different external solicitations which are highly turbulent and uncertain. 
Among them, the \textit{metocean} (abbreviation of meteorological and oceanography) environmental conditions play an important role.
Other types of solicitations affect the exploitation of offshore wind turbines (e.g., corrosion of the structure, global scour, marine growth, stress concentration factor induced by the manufacturing quality). 

In this context, numerical models have been developed to certify the structural integrity of OWTs with respect to their solicitations.
A wind farm project planned at given location should pass different validation procedures established by international standards such as the \elias{add full IEC name} \citet{iec_2019}
As wind turbine structures face a large amount of stress cycles in their lifetime (up to $10^8$ for 25 years of operation), this chapter will particularly focus on fatigue damage assessment.


This chapter presents a brief introduction to wind turbine modeling and design, structure as follows: 
Section \ref{sec:metocean_simulation} presents the methods used for wind and wave generation and wake simulation at a farm scale; 
Section \ref{sec:owt_modeling} recalls elements of theory associated with wind turbine modeling; 
Section \ref{sec:owt_design} introduces recommended practices regarding design and operation;
finally, Section \ref{sec:owt_uncertainties} gives a description of the various sources of uncertainties considered in this thesis. 
Considering the standard uncertainty quantification diagram presented in \elias{add ref}, the material of this chapter is related to the step A (problem specification) and the step B (uncertainty quantification). 
%To go further this general introduction, the reader might refer to the following sources on the different topics 



%============================================================%
%============================================================%
\section{Metocean conditions simulation} \label{sec:metocean_simulation}
%============================================================%
%============================================================%


Heating differences of the surface by the sun
Differences in atmospheric pressure generate winds.
Winds generally blow from high-pressure areas to low-pressure areas.
The Coriolis effect, related to Earth's rotation, causes some winds to travel along the edges of the high-pressure and low-pressure systems.

The wind is a highly variable resource, making its exploitation for energy production uncertain.
This variability is both expressed in space and time with different behaviors depending on the scales studied. 



\begin{itemize}
    \item Why is there wind? It's a complex phenomenon related to many effects such as the rotation of the earth, Coriolis effect \elias{add sources}
    \item Regarding the large timescale, yearly seasonal fluctuations of wind conditions are well-defined and predictable
    \item Prediction at a shorter timescale are usually unreliable above a few days ahead. 
    \item Below a few days, the wind variation can be described the wind power spectral density. (Van der Hoven spectrum plot and explaination)
    \item We commonly work on 10min intervals to capture the turbulent behavior
    \item Warning: this spectrum pattern is not guaranteed at any given location. 
    \item In fact, the wind variations are strongly impacted by the surface roughness, topology, thermal effects 
    \item Opening: global warming effect on the long term metocean conditions, mostly affecting the frequency and intensity of extreme events.
\end{itemize}

\elias{Add something general regrading wave modeling}





%============================================================%
\subsection{Turbulent wind generation}
%============================================================%
\elias{Check the TurbSim documentation to understand the numerical process}
\elias{Check the paper on turbulence models for a synthetic introduction of spectral models and coherence functions}

\begin{itemize}
    \item Vertical wind profile
    \item Kaimal/Mann spectral methods for generation
    \item exponential spatial coherance methods
    \item Sandia method allowing to generate the time series independently
    \item Wind direction? 
\end{itemize}


\begin{itemize}
    \item Amazing visual idea: at multiple 10-minutes periods, one can generate different wind speeds  
\end{itemize}

\elias{Opening: vortex modeling for wind turbines Emmanuel Branlard thesis}

%============================================================%
\subsection{Irregular wave generation}
%============================================================%
\elias{Milano section 1.4}


%============================================================%
\subsection{Wake modeling}
%============================================================%
\elias{Read the chapter 9 from Wind Energy handbook + check the paper with IFPEN}





%============================================================%
%============================================================%
\section{Wind turbine multi-physics modeling} \label{sec:owt_modeling}
%============================================================%
%============================================================%

Offshore wind turbine models are coupling multiple physics such as aerodynamics, hydrodynamics, mechanical elasticity, control and mooring dynamics for floating OWT. 
Similarly to the usual practices from the offshore oil \& gas industry, OWT have been first modeled in the frequency domain. 
At an early design stage, a study in the frequency domain gives a rough idea of the system's feasibility by computing its natural frequencies.
An OWT should not have its natural frequencies in the same range as the main frequencies of the wave energy spectra. 
Otherwise, such systems can be subject to critical dynamic resonance, leading to their failure.

Beyond this preliminary check, frequency-domain approaches present limits for OWT modeling. 
As they rely on linear assumptions, they are unable to model the non-linearities and transient loading phases \citep{matha_2011_ISOPE}. 
These aspects happen to be essential in the design of OWT \citep{jonkman_2011_ISOPE}.
As an alternative, the behavior of OWT systems are also simulated in the time domain. 

\elias{In the time domain, such systems may be models at different fidelities.
The diagram from Veer, illustrates the increasing complexities two physics involved in OWT modeling (aerodynamics and structural dynamics).
The model studied in our case is low fidelity, allowing to perform uncertainty quantification around it.} 

\elias{At the wind turbine scale, the numerical model studied in this work is actually a chain of three models executed sequentially.} 

\elias{At the wind farm scale, the wake effect also plays an important role.} 



%============================================================%
\subsection{Aerodynamics of horizontal axis wind turbines}
%============================================================%


%============================================================%
\subsection{Hydrodynamics}
%============================================================%


%============================================================%
\subsection{Command and control}
%============================================================%


%============================================================%
\subsection{Structural dynamics}
%============================================================%


%============================================================%
\subsection{Fatigue damage}
%============================================================%





%============================================================%
%============================================================%
\section{Design and operation practices} \label{sec:owt_design}
%============================================================%
%============================================================%

Generally speaking, the energy available in the wind is proportional to the cube of the wind speed, given the well-known power production equation:
\begin{equation}
    P = \frac12 C_p \rho A U^3,
\end{equation}  
where \elias{continue the description}.


%============================================================%
\subsection{Wind turbine design and operation}
%============================================================%
    \begin{itemize}
        \item Different designs : soft-stiff, stiff-stiff 
        \item Design load cases 
        \item How does it work : different ranges of function depending on the wind speed (cut-in, cut-off)
        \item 
    \end{itemize}

%============================================================%
\subsection{Further design considerations}
%============================================================%
\begin{itemize}
    \item Soil modeling
    \item Marine growth
    \item Global scour
    \item Port logistics
    \item Grid impact
    \item Environmental impact and social acceptance 
    \item Manufacturing quality inducing stress concentration factor
\end{itemize}


%============================================================%
\subsection{Maintenance and end-of-life management}
%============================================================%
\begin{itemize}
    \item Operation and management
    \item Repowering vs. revamping
    \item Recycling 
\end{itemize}


%============================================================%
%============================================================%
\section{Uncertain inputs} \label{sec:owt_uncertainties}
%============================================================%
%============================================================%

%============================================================%
\subsection{Environmental inputs}
%============================================================%


%============================================================%
\subsection{System inputs}
%============================================================%


%============================================================%
\subsection{Probabilistic fatigue assessment}
%============================================================%




%============================================================%
%============================================================%
\section{Conclusion}
%============================================================%
%============================================================%