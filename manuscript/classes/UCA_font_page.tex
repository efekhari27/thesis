{
\thispagestyle{empty}
% Header
\begin{figure}[t]
    \vspace*{-2.7cm}
    \begin{adjustwidth}{-28mm}{-21mm}
      \includegraphics[width=\paperwidth]{./part1/figures/UCA_STIC.jpg}
    \end{adjustwidth}
  \end{figure} 
% Title
%\vspace*{5pt}

\begin{adjustwidth}{-8mm}{-8mm}
\begin{center}
  {\setstretch{2.}
  {\Huge \scshape{Quantification d'incertitudes en simulation multiphysique pour la gestion d’actifs \'{e}oliens}}\\\vfill
  {\Large \textbf{Elias FEKHARI}}}\\
  {\large EDF R\&D, Laboratoire d'Informatique, Signaux et Systèmes de Sophia-Antipolis (I3S), \\Universit\'{e} Nice Côte d'Azur}
\end{center}
%
\vfill
\noindent
Pr\'{e}sent\'{e}e en vue de l’obtention du grade de docteur en Automatique, traitement du signal et des images 
d’Universit\'{e} Côte d’Azur, dirig\'{e}e par Bertrand Iooss, soutenue le 12 mars 2024.\\

\noindent
\textbf{Devant le jury compos\'{e} de~:}\\
\begin{tabular}{llll}
  Directeur       & Bertrand IOOSS      & Chercheur senior & EDF R\&D, Chatou \\
  Co-encadrant    & Vincent CHABRIDON   & Ing\'{e}nieur-chercheur        & EDF R\&D, Chatou \\
  Rapporteurs     & Franck SCHOEFS      & Professeur des universit\'{e}s & Nantes Universit\'{e}, Nantes\\
                  & Daniel STRAUB       & Professeur des universit\'{e}s & TUM, Munich\\
  Examinateurs    & Mireille BOSSY      & Directrice de recherche    & INRIA, Sophia-Antipolis\\
                  & S\'{e}bastien DA VEIGA  & Professeur associ\'{e}         & ENSAI, Rennes\\
                  & Bruno SUDRET        & Professeur des universit\'{e}s & ETH, Zurich \\ 
  Invit\'{e}s         & Ana\"is LOVERA      & Ing\'{e}nieure-chercheur       & EDF R\&D, Saclay\\
                  & Joseph MUR\'E       & Ing\'{e}nieur-chercheur        & EDF R\&D, Chatou \\
\end{tabular}
\end{adjustwidth}

%\vfill
%\begin{center}
%  \includegraphics[width=0.25\textwidth]{classes/logo_EDF.png}
%\end{center}
%}