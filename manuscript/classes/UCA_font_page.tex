{
\thispagestyle{empty}
% Header
\begin{figure}[t]
    \vspace*{-2.7cm}
    \begin{adjustwidth}{-28mm}{-21mm}
      \includegraphics[width=\paperwidth]{./part1/figures/UCA_STIC.jpg}
    \end{adjustwidth}
  \end{figure} 
% Title
%\vspace*{5pt}

\begin{adjustwidth}{-8mm}{-8mm}
\begin{center}
  {\setstretch{2.}
  {\Huge \scshape{Quantification d'incertitudes en simulation multiphysique pour la gestion d’actifs éoliens}}\\\vfill
  {\Large \textbf{Elias FEKHARI}}}\\
  {\large EDF R\&D, Laboratoire d'Informatique, Signaux et Systèmes de Sophia-Antipolis (I3S), \\Université Nice Côte d'Azur}
\end{center}
%
\vfill
\noindent
Présentée en vue de l’obtention du grade de docteur en Automatique, traitement du signal et des images 
d’Université Côte d’Azur, dirigée par Bertrand Iooss, soutenue le 12 mars 2024.\\

\noindent
\textbf{Devant le jury composé de~:}\\
\begin{tabular}{llll}
  Directeur       & Bertrand IOOSS      & Chercheur senior & EDF R\&D, Chatou \\
  Co-encadrant    & Vincent CHABRIDON   & Ingénieur-chercheur        & EDF R\&D, Chatou \\
  Rapporteurs     & Franck SCHOEFS      & Professeur des universités & Nantes Université, Nantes\\
                  & Daniel STRAUB       & Professeur des universités & TUM, Munich\\
  Examinateurs    & Mireille BOSSY      & Directrice de recherche    & INRIA, Sophia-Antipolis\\
                  & Sébastien DA VEIGA  & Professeur associé         & ENSAI, Rennes\\
                  & Bruno SUDRET        & Professeur des universités & ETH, Zurich \\ 
  Invités         & Ana\"is LOVERA      & Ingénieure-chercheur       & EDF R\&D, Saclay\\
                  & Joseph MUR\'E       & Ingénieur-chercheur        & EDF R\&D, Chatou \\
\end{tabular}
\end{adjustwidth}

%\vfill
%\begin{center}
%  \includegraphics[width=0.25\textwidth]{classes/logo_EDF.png}
%\end{center}
%}