\cleardoublepage
\setsinglecolumn
\chapter*{\centering \Large R\'{e}sum\'{e}}
\thispagestyle{empty}

{
\footnotesize
L'\'{e}olien en mer est l'un des moyens de r\'{e}duire la part des \'{e}nergies fossiles dans le mix \'{e}lectrique mondial. 
Cette technologie b\'{e}n\'{e}ficie de vents plus r\'{e}guliers que l'\'{e}olien terrestre, principalement en raison de l'absence de relief. 
L'exploitation en mer permet \'{e}galement d'installer des \'{e}oliennes plus grandes et plus puissantes, ce qui pose plusieurs problèmes concernant la logistique portuaire, les besoins en ressources naturelles rares et les processus de recyclage en fin de vie.

Les \'{e}oliennes en mer sont des systèmes dynamiques qui interagissent avec un environnement fortement incertain. 
Le traitement des incertitudes autour de leurs modèles de simulation num\'{e}rique est donc essentiel pour proposer une conception et une exploitation fiable. 
Cependant, l'application d'une quantification d'incertitudes à ces systèmes soulève de nombreuses probl\'{e}matiques et n\'{e}cessite le couplage de donn\'{e}es avec des modèles num\'{e}riques multiphysiques.

Cette thèse traite d'abord des problèmes li\'{e}s à la mod\'{e}lisation probabiliste multivari\'{e}e des conditions environnementales en mer. 
Une approche semi-param\'{e}trique est propos\'{e}e, m\'{e}langeant des m\'{e}thodes param\'{e}triques pour l'ajustement des marginales avec la copule empirique de Bernstein pour approximer la structure de d\'{e}pendance complexe entre les variables environnementales. 
Après avoir d\'{e}fini un modèle probabiliste des conditions environnementales, les perturbations caus\'{e}es par l'effet de sillage des turbines sont \'{e}tudi\'{e}es à l'\'{e}chelle de la ferme en cr\'{e}ant des groupes de turbines pr\'{e}sentant des perturbations similaires. 
Ce regroupement pr\'{e}liminaire vise à r\'{e}duire le nombre d'\'{e}tudes de chargement à l'\'{e}chelle d'une ferme (par exemple, pour l'\'{e}valuation de la fatigue).

Le deuxième axe m\'{e}thodologique de ce travail concerne la propagation d'incertitudes pour l'estimation de tendances centrales ou d'\'{e}v\'{e}nements rares. 
Comme alternative aux cas de charge recommand\'{e}s par les normes internationales pour le calcul du dommage en fatigue moyen, la m\'{e}thode du kernel herding s'est av\'{e}r\'{e}e être une solution efficace et flexible pour r\'{e}aliser ce calcul à partir de donn\'{e}es (c'est-à-dire en sous-\'{e}chantillonnant des donn\'{e}es environnementales sans inf\'{e}rence). 
Cette m\'{e}thode de quadrature Bay\'{e}sienne est \'{e}galement adapt\'{e}e à la construction de bases de test pour l'estimation de la pr\'{e}cision moyenne de modèles d'apprentissage statistique.

Pour estimer des \'{e}v\'{e}nements rares, une nouvelle m\'{e}thode int\'{e}grant une copule non param\'{e}trique dans un m\'{e}canisme d'\'{e}chantillonnage pr\'{e}f\'{e}rentiel adaptatif est propos\'{e}e. 
Cette approche donne des r\'{e}sultats \'{e}quivalents à la m\'{e}thode de \textit{subset simulation} tout en \'{e}vitant un \'{e}chantillonnage de Monte Carlo par chaîne de Markov. 
Par la suite, la fiabilit\'{e} en fatigue d'une \'{e}olienne en mer est \'{e}tudi\'{e}e en tenant compte des incertitudes li\'{e}es à l'environnement, mais aussi celles d'autres variables comme la rigidit\'{e} du sol, l'erreur d'alignement de la nacelle, la courbe de W\"{o}hler et la r\'{e}sistance critique à la fatigue. 
La robustesse de la fiabilit\'{e} estim\'{e}e est ensuite \'{e}tudi\'{e}e à l'aide des \textit{perturbed-law based indices}. 
Enfin, pour assurer la reproductibilit\'{e} des r\'{e}sultats num\'{e}riques, la plupart des d\'{e}veloppements pr\'{e}sent\'{e}s dans ce travail sont open source et document\'{e}s.

\vspace*{1cm}
\noindent
\textbf{Mots cl\'{e}s :} quantification d'incertitudes, plans d'exp\'{e}riences, faibilit\'{e} des structures, endommagement en fatigue, \'{e}nergie \'{e}olienne.
}