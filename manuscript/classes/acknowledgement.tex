% ************************** Thesis Acknowledgements **************************

\cleardoublepage
\setsinglecolumn
\chapter*{\centering \Large Remerciements}
\thispagestyle{empty}

{
\footnotesize
Je dédie ces lignes aux personnes qui ont contribué, de près ou de loi, à la réalisation de cette thèse. 
Ces travaux n'auraient pas été possibles sans les conseils, les enseignements, et le soutien de nombreuses personnes que j'ai pu côtoyer au cours de mon parcours. 

J'ai eu l'honneur de défendre ma thèse devant un jury remarquable. 
Tout d'abord, j'adresse mes sincères remerciements aux rapporteurs Messieurs Daniel \textsc{Straub} et Franck \textsc{Schoefs} pour la qualité de votre relecture, la pertinence de vos remarques et vos rapports très encourageants. 
Je remercie également la présidente du jury, Madame Mireille \textsc{Bossy} pour votre bienveillance et la conduite de la soutenance qui a personnellement été un exercice très instructif. 
Je tiens ensuite à remercier chaleureusement les examinateurs du jury, Messieurs Bruno \textsc{Sudret} et Sébastien \textsc{Da Veiga} pour votre disponibilité, pour toutes nos discussions enrichissantes ainsi que vos nombreuses recommandations de lecture durant ces trois ans.   

Pendant cette thèse, j'ai eu la chance d'être encadré par un trio très complémentaire auprès duquel j'ai énormément appris, je vous remercie d'avoir accepté de m'accompagner dans cette aventure. 
Comme d'habitude, tout à commencé par des discussions avec Vincent \textsc{Chabridon}, qui m'a partagé sa passion pour la recherche et a réveillé mes anciennes envies de thèse. 
En plus d'avoir renforcé notre amitié, je te remercie immensément pour tes innombrables idées et ton implication constante qui m'a fait grandir. 
Ensuite, cette thèse n'aurait pas eu lieu sans la confiance que Bertrand \textsc{Iooss} m'a accordé dès la présentation du sujet.   
Je te remercie profondément pour cette direction hors pair, pour toutes les opportunités que tu m'as donné et ta disponibilité pour transmettre ton expérience. 
Enfin, Joseph \textsc{Muré} a été d'une aide précieuse pour approfondir et implémenter certains concepts mathématiques. 
Je te remercie grandement pour ta pédagogie, ton enthousiasme à toute épreuve et les discussions politiques passionnantes. 

J'ai eu le plaisir pendant ces trois ans d'interagir et apprendre auprès de beaucoup de rechercheurs. 
Durant ma première année, nous avons eu la chance de travailler avec Luc P. et Maria-Jo\~ao R., merci à eux pour cette expérience très enrichissante qui a ouvert plusieurs pistes méthodologiques. 
Une partie de mes travaux de thèse a co\"incidé avec la contribution d'EDF R\&D au projet européen HIPERWIND. 
Cette collaboration entre des partenaires académiques et industriels m'a permi de confronter mes travaux à différents points de vue. 
Je remercie tous les contributeurs de ce projet, notamment Erik V., Stefano M., Nikolay D., Martin G., Alexis C., Miguel M.Z., Styfen S., Baptiste J. et Fabien R. 
Je tiens particulièrement à remercier certaines personnes en charge du développement du code DIEGO, comme Matteo C. pour sa curiosité, Christophe P. pour son engouement, et enfin Anaïs L. pour son optimisme et son aide tout au long du projet HIPERWIND. 
J'espère que ces travaux ouvreront la voie vers d'autres collaborations scientifiques internes entre la modélisation éolienne et la quantification des incertitudes. 


Plus largement, je souhaite remercier plusieurs personnes impliquées dans la communauté du RT-UQ (ex. GDR Mascot-Num). 
À commencer par les enseignants de Sigma Clermont qui m'ont initié à la quantification d'incertitudes, en particulier Jean-Marc B., Nicolas G., Pierre B. et Cécile M.  
Je remercie également les nombreux collègues avec qui j'ai eu l'occasion de partager plusieurs écoles d'été et conférences tels que 
Merlin K., Julien P., Antoine A., Marouane I., Nicolas B., Jérôme S., \'Alvaro R.D., Guillaume C., Edgar J., Baalu K.,
Clément G., Amadine M., Rudy C., Guillaume D., Gabriel S., Clément B., Brian S., Ga\"el P., Charlie S., Babacar S., Adrien S., etc. 

EDF R\&D est un endroit privilégié pour aborder des problématiques industrielles uniques aux côtés de nombreux collègues chevronnés.
Je remercie le management du département PRISME pour m'avoir donné l'opportunité de réaliser cette thèse, en particulier Soufien H. pour sa compréhension.  
Depuis mon arrivée à Chatou en 2017, j'ai été accueilli par une équipe dynamique, avec des profils très divers. 
Je tiens entre autre à remercier Emmanuel A. pour sa confiance depuis mon stage, 
Emmanuel R., dit Manu, pour son grand professionnalisme et son mentorat bienveillant, 
Jérôme L. pour son aide précieuse sur les sujets de gestion d'actifs, 
Mich\"el B. pour son accompagnement sur OpenTURNS, 
Thibault D. et Carole M. pour avoir défendu mon projet de thèse, 
Roman S., Mohammed M. et Brahim M., pour avoir été d'excellents colocataires de bureau, 
Charles D. pour toutes les inspirations musicales et les cours de batterie, 
Pablo P.A. pour avoir été mon camarade depuis notre arrivée en stage. 

Sur un plan plus personnel, j'ai toujours été très soutenu dans cette expérience. 
À commencer par tous mes amis, que j'ai trop peu solliciter durant ces trois ans, comme le groupe des GZ de L'IFMA, les anciens de Mermoz, et les derniers compères du Crès comme Martin L.
Ensuite, je tiens à dédier ces travaux à toute ma famille que j'aime en France, au Maroc, au Portugal et en Slovénie, en particulier à l'incroyable équipe de cousins avec laquelle j'ai grandi, 
Titima, Adam, Siham, Amine, Meryem, Kenza, Zakaria, Sma\"il et Elias junior. 
Pour conclure ces remerciements, je souhaite profondément remercier mon frère Idris et ma sœur Inès pour leurs encouragements constants, 
mes parents à qui je dois tout, qui ont toujours cru en moi et motivé pour me dépasser. 
Enfin, je souhaite remercier infiniment ma partenaire Zala, pour ta patience et ton soutien inconditionnel malgré tous les sacrifices et mes longues soirées passées à travailler. 




}

