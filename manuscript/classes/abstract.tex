% ************************** Thesis Abstract *****************************
% Use `abstract' as an option in the document class to print only the titlepage and the abstract.
%4000 caracters
\begin{abstract}
\footnotesize
Offshore wind energy is one of the ways to reduce the share of fossil fuels in the global electricity mix. 
This technology benefits from more consistent winds than the onshore one, mainly due to the absence of terrain roughness. 
Operating offshore also allows the installation of larger and more powerful wind turbines, which poses several scaling issues about port logistics, the demand for critical natural resources, and sustainable end-of-life processes.
 
Offshore wind turbines are dynamic systems interacting with a highly uncertain environment. 
Uncertainty quantification of the multi-physics numerical models used to simulate them is therefore essential to propose risk-informed design and operation. 
However, developing a dedicated uncertainty quantification strategy for these systems raises numerous questions and requires coupling data with multi-physics numerical models.
 
This thesis first addresses the problems related to the multivariate probabilistic modeling of offshore environmental conditions. 
A hybrid approach is suggested, mixing parametric methods for marginals fitting with the empirical Bernstein copula to fit the complex dependence structure among environmental variables. 
After defining a probabilistic model of the ambient metocean conditions, the perturbations caused by the turbines' wake effect are studied at the farm scale by creating clusters of similarly perturbed turbines. 
This preliminary clustering aims to reduce the number of loading studies at the farm scale (e.g., for fatigue assessment).
 
The second methodological axis of this work concerns uncertainty propagation for both central study and rare event estimation. 
As an alternative to the design load cases recommended by international standards for the mean cumulative damage estimation, the kernel herding method proved to be an efficient and flexible solution for given-data uncertainty propagation (i.e., directly subsampling from a large dataset without inference). 
This Bayesian quadrature method is also well-suited for the construction of test samples for the estimation of the mean predictivity of statistical learning models.
 
For rare event estimation, a new method incorporating a nonparametric copula into an adaptive importance sampling mechanism is proposed. 
This method displays equivalent results to splitting methods as the subset simulation while avoiding any Markov Chain Monte Carlo sampling and thus generating independent and identically distributed samples. 
Then, the fatigue reliability of an offshore wind turbine is studied with respect to uncertain environmental variables while considering other variables related to soil stiffness, yaw misalignment, stress-number of cycles curve, and critical damage resistance. 
The robustness of the estimated reliability is then studied using perturbed-law based indices. 
Finally, to ensure the reproducibility of the numerical results, most of the developments presented in this work are open source and documented.
\end{abstract}

\vspace*{1cm}
{
\footnotesize
\noindent
\textbf{Keywords.} Uncertainty quantification, design of experiments, reliability analysis, fatigue damage, wind energy. 
}